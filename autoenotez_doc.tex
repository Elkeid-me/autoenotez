%% autoenotez_doc.tex
%% Copyright 2023 Elkeid Me
%
% This work may be distributed and/or modified under the
% conditions of the LaTeX Project Public License, either version 1.3
% of this license or (at your option) any later version.
% The latest version of this license is in
%   http://www.latex-project.org/lppl.txt
% and version 1.3c or later is part of all distributions of LaTeX
% version 2008 or later.
%
% This work has the LPPL maintenance status `maintained'.
%
% The Current Maintainer of this work is Elkeid Me.
%
% This work consists of the file autoenotez.sty, and its
% documentation autoenotez_doc.tex and 2.tex.

\PassOptionsToPackage{no-math}{fontspec}
\documentclass[a4paper]{ctexart}
\usepackage{autoenotez}

\title{\textsf{autoenotez} 手册}
\author{Elkeid Me}
\date{\today v\autoenotezversion}

\begin{document}
    \maketitle

    \section{简介}

    \textsf{autoenotez} 旨在分离尾注/脚注的标记与内容。类似 \textsf{sepfootnotes} 包的功能。

    \textsf{autoenotez} 包是在排版某本书时编写的。

    \section{示例}

    首先,你需要在一个新的章/节后使用 \verb|\NewEndnotes|,以设置 \textsf{autoenotez} 内部的一些寄存器;

    \begin{verbatim}
\section{一一四五一四章}
\NewEndnotes
    \end{verbatim}

    然后,在 \verb|\NewEndnotes| 后,使用 \verb|\EndnoteContent| 按顺序给出本章/节用到的尾注/脚注内容:

    \begin{verbatim}
\section{一一四五一四章}
\NewEndnotes
\EndnoteContent{尾注 1}
\EndnoteContent{尾注 2}
    \end{verbatim}

    接下来,在正文中,以 \verb|\Endnote| 标记出尾注的位置

    \begin{verbatim}
\section{一一四五一四章}
\NewEndnotes
\EndnoteContent{尾注 1}
\EndnoteContent{尾注 2}
114514\Endnote 1919810\Endnote
    \end{verbatim}

    最后,使用 \verb|\printendnotes| 打印尾注。你可能需要多次编译以达到效果。完整的样例如下:

    \begin{verbatim}
\documentclass{ctexart}
\begin{document}
    \section{一一四五一四章}
    \NewEndnotes
    \EndnoteContent{尾注 1}
    \EndnoteContent{尾注 2}
    114514\Endnote 1919810\Endnote
    \printendnotes
\end{document}
    \end{verbatim}

    为了达到脚注标记与内容的真正分离,所有的 \verb|\EndnoteContent| 也可以挪到一个单独的文件中,并以 \verb|\input| 引用:

    \verb|1.tex|
    \begin{verbatim}
\documentclass{ctexart}
\begin{document}
    \section{一一四五一四章}
    \NewEndnotes
    %% 2.tex
%% Copyright 2023 Elkeid Me
%
% This work may be distributed and/or modified under the
% conditions of the LaTeX Project Public License, either version 1.3
% of this license or (at your option) any later version.
% The latest version of this license is in
%   http://www.latex-project.org/lppl.txt
% and version 1.3 or later is part of all distributions of LaTeX
% version 2023/09/10 or later.
%
% This work has the LPPL maintenance status `maintained'.
%
% The Current Maintainer of this work is Elkeid Me.
%
% This work consists of the file autoenotez.sty, and its
% documentation autoenotez_doc.tex and 2.tex.

\EndnoteContent{尾注 1}
\EndnoteContent{尾注 2}

    114514\Endnote 1919810\Endnote
    \printendnotes
\end{document}
    \end{verbatim}

    \verb|2.tex|
    \begin{verbatim}
\EndnoteContent{尾注 1}
\EndnoteContent{尾注 2}
    \end{verbatim}

    而 \verb|\NewEndnotes| 提供了可选参数,即 \verb|\NewEndnotes %% 2.tex
%% Copyright 2023 Elkeid Me
%
% This work may be distributed and/or modified under the
% conditions of the LaTeX Project Public License, either version 1.3
% of this license or (at your option) any later version.
% The latest version of this license is in
%   http://www.latex-project.org/lppl.txt
% and version 1.3 or later is part of all distributions of LaTeX
% version 2023/09/10 or later.
%
% This work has the LPPL maintenance status `maintained'.
%
% The Current Maintainer of this work is Elkeid Me.
%
% This work consists of the file autoenotez.sty, and its
% documentation autoenotez_doc.tex and 2.tex.

\EndnoteContent{尾注 1}
\EndnoteContent{尾注 2}
| 可以直接写成 \verb|\NewEndnotes[2.tex]|。

    实际的结果如下页:
    \newpage

    \section{一一四五一四章}

    \NewEndnotes[2.tex]

    114514\Endnote 1919810\Endnote

    \printendnotes

    \verb|\NewFootnotes|、\verb|\Footnote| 与 \verb|\FootnoteContent| 的用法类似,只不过是负责尾注的。\textbf{不需要} \verb|\printendnotes|

    备注:尾注的标题是 Notes 怎么办?其实这是 \textsf{enotez} 包的设定,可以通过 \verb|\setenotez| 修改。阅读 \verb|enotez| 文档了解详细信息。
\end{document}
